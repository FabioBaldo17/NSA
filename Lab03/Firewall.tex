\section{Configuration}

\subsection{Question P1}
In our case having a static or dynamic routing is not important beacuse all the traffic passes through the firewall.

\subsection{Question P2}
Thanks to the commands \scom{show run object}  and \scom{show run nat} the following output has been obtained. This configuration is needed for translating the ip address such that when an inside of the specific subnet, their IPs wil be translated accordingly to the interface.
\href{https://www.cisco.com/c/en/us/td/docs/security/asa/asa96/configuration/firewall/asa-96-firewall-config/nat-reference.html}{manual}
\inputminted{text}{ASA_P2.txt}
\Caption{output for question P2}
\label{conf:Q_P2}


\subsection{Question P3}
Only the IPs of the subnet \Lcode{192.168.10.10 255.255.255.0} will be translated to IPs of the interface \Lcode{interface GigabitEthernet0/0}

\subsection{Question P4}
% TODO cannot validate because no GUI present 
\subsection{Question P5}
Create a username with password:
\mscom{ASA(config)\# username cisco password cisco}

Configure this local username to authenticate with SSH:
\mscom{ASA(config)\# aaa authentication ssh console LOCAL}

Configure this local username to authenticate with Telnet:
\mscom{ASA(config)\# aaa authentication telnet console LOCAL}
Create RSA key pair:
\mscom{ASA(config)\# crypto key generate rsa modulus 2048}

Now specify only particular hosts or network to connect to the device using SSH:
from the inside
\mscom{ASA(config)\# ssh 192.168.10.0 255.255.255.0 inside}
from the outside
\mscom{ASA(config)\# ssh 172.16.3.10 255.255.255.255 outside}

as a consequence of the procedure it is possible to connect via ssh to the ASA from HOST B and C

\subsection{Question P6}

\subsection{Question P7}
The commands used for testing the communication between the two hosts have been:
\mscom{HostA\#nc -n -l -p 80 -v}
\mscom{HostC\#nc -v 209.165.200.227 80}

\subsection{Question P8}
For testing the connection from the HostB to the hostC the IP had not to be changed because no rules are applied from the inside to the dmz.

\subsection{Question P9}
The results of the scans are the followings:

\inputminted{text}{nmap_HC_sA.txt}
\Caption{Nmap from HostC}
\label{conf:Q_P2}

\subsection{Question P10}


\subsection{Question P11}
problems with java WS

\subsection{Question P12}
problems with java WS